\section{Block Product, Error Detection and Correction}
\label{sec:block}

\subsection{Computing checksums}

The checksums are computed by a single kernel, that based on a parameter computes row or column checksums.

For row checksums, a thread block is generated for each row, with a set of threads that initially sum a portion of the row, in a linear way.
After that, the different threads of the block organize themselves to sum all the partial sums in a dichotomous way.

Similarly, column checksums generate a thread block for each column, performing the same operations in the orthogonal direction.

After the computation, the kernel is able to store the checksum either in the last row or column of the matrix, or into a separate vector, in order to avoid overriding the previous checksums (the control checksums of C).

\img{checksums-pseudocode}{0.7\linewidth}{the pseudocode of the kernel that computes the checksums}

\subsection{Computing the product}

When the matrices are small enough to fit on GPU (or we are working on blocks), the multiplication is realized by means of the standard tiled algorithm for GPU.

Initially, we believed that we could apply one or two recursions of Strassen algorithm to reduce the number of products.
Then, we realized that this would make the same error appear in different submatrices: this would make it much harder to correct it, since non-collinear errors are not correctable.
For this reason, we decided to drop Strassen algorithm at this smaller size level.

\img{tiled-matmul-pseudocode}{0.7\linewidth}{the pseudocode of the kernel that computes the matrix multiplication using the tiled approach}

\subsection{Error addition}

In order to demonstrate the error correction capability, some errors can be added after the computation of the product.
The number of errors per product can be set by the user, as well as if they must be collinear or not.

The program randomly chooses a set of distinct elements where to add errors, and adds a random delta to the chosen element.

\subsection{Error correction}

After computing the product and adding the errors, checksums are computed again on the result, both on the rows and the columns.
A kernel is then run to find mismatches between the control checksums (the ones obtained as the sum of rows or columns of C) and the ones in C (obtained from the multiplication).

The mismatches are then compared to check if errors are present, and if they are collinear (otherwhise, they are not correctable).
If the errors are correctable, they are corrected by using the control checksums.
If they are not correctable, or there are errors on the checksums, then the caller is notified.

\img{edc-pseudocode}{0.7\linewidth}{the pseudocode of the function that detects and corrects the errors (\texttt{find\_mismatches} is a GPU kernel, while the rest is executed on CPU)}
