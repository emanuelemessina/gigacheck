\section{Introduction}

Matrix multiplication is a fundamental operation
in various scientific, engineering, and data-driven domains,
including deep learning, scientific simulations, and aerospace systems.
These fields often rely on high-performance hardware accelerators, such as GPUs,
to handle computationally intensive tasks.
However, the complexity and high utilization of modern GPU environments
make them susceptible to errors caused by hardware faults,
thermal fluctuations, and cosmic rays, particularly in aerospace applications
where hardware is exposed to extreme environments.
Fault-tolerant computation strategies are critical in such contexts,
as undetected errors in matrix operations can lead to model mispredictions
in neural networks or potentially catastrophic failures in mission-critical
aerospace systems.
In this report, we present our implementation of a fault-tolerant
matrix multiplication algorithm on CUDA, capable of handling matrices whose size would not allow a single-shot computation of the multiplication inside the GPU, as it would require an excess with respect to the available global memory.

This paper is structured as follows: TODO METTERE LINK ALLE SEZIONI
\begin{itemize}
  \item In SECTION 2 we discuss the background theory regarding matrix multiplication and ABFT.
  \item In SECTION 3 we present an overview of our algorithm to see how it works in the bigger picture, before going into the details in the subsequent sections.
  \item In SECTION 4 we explain how we perform the block level product, error detection and correction.
  \item In SECTION 5 we illustrate our buffering strategies when dealing with large matrices.
  \item In SECTION 6 we discuss the performance differences when varying parameters.
\end{itemize}
