\section{Background Theory}
\label{sec:background}

\subsection{ABFT matrix multiplication}

The basic idea behind Algorithm-Based Fault Tolerant (ABFT) matrix multiplication
is to use checksums to verify the correctness of the computation.
The following paragraphs illustrate the steps we perform:

\subsubsection{Checksum Calculation}
Let \( A \) be an \( m \times n \) matrix,
\( B \) be an \( n \times q \) matrix,
and \( C \) be the resulting \( m \times q \) matrix
from the multiplication \( C = A B \).
Before performing the matrix multiplication,
we calculate the column checksum of \( A \) and the row checksum of \( B \):

\[
  \textnormal{Column checksum of } A: \quad
  \mathbf{c}_A =
  \left[
    \sum_{i=1}^{n} a_{i1},
    \sum_{i=1}^{n} a_{i2},
    \ldots, \sum_{i=1}^{n} a_{in}
  \right]
\]

\[
  \textnormal{Row checksum of } B: \quad
  \mathbf{r}_B = \left[ \sum_{j=1}^{n} b_{1j}, \sum_{j=1}^{n} b_{2j}, \ldots, \sum_{j=1}^{n} b_{nj} \right]^T
\]

\textit{
  The row checksum of a matrix is a column vector where each element
  is the sum of the elements in the corresponding row of the matrix.
  Similarly, the column checksum is a row vector where
each element is the sum of the elements in the corresponding column of the matrix.}

\subsubsection{Augmented Matrices}

We then create augmented matrices $A_{c}$ (\( m+1 \times n \)) and $B_{r}$ (\( n \times q+1 \))
by appending the column and row checksum vectors, respectively, to the original matrices:
Thus we get:
\[
  A_{c} =
  \left[
    \begin{array}{c}
      A \\
      \midrule
      \mathbf{c}_A
    \end{array}
  \right]
  \quad
  ,
  \quad
  B_{r} = \left[
    \begin{array}{c|c}
      B &
      \mathbf{r}_B
    \end{array}
  \right]
\]
\subsubsection{Multiplication}
We multiply $ A_{c} $ and $ B_{r} $ together, yielding
\[
  C_{cr} = A_{c} B_{r} = \left[
    \begin{array}{c|c}
      A B               & A \, \mathbf{r}_B            \\
      \hline
      \mathbf{c}_A \, B & \mathbf{c}_A \, \mathbf{r}_B
    \end{array}
  \right]
\]
We see that the original product is preserved
in the upper left block,
while the other blocks contain checksum information.

\subsubsection{Checksum Verification}
Considering the column and row checksums of the upper left block of
the resulting matrix \( C_{cr} \), the following properties hold:

\[
  \mathbf{c}_{AB} = \left[\mathbf{c}_A \, B \right]
  \quad
  ,
  \quad
  \mathbf{r}_{AB} = \left[A \, \mathbf{r}_B \right]
\]

The proof is trivial.

This is the property that we ultimately exploit to correct errors in the computation,
because if the upper block returned to us is corrupted,
then we can compute its checksums (we will call them control checksums)
and compare them against the peripheral blocks to at least detect the corruption of the result.

In general, the column checksum of the upper blocks of the augmented result
is equal to the last row of the augmented result, and the row cheksum of the left blocks of the augmented result
is equal to the last column the augmented result. In formulas:

\[
  \mathbf{c}_{\text{control}} = \mathbf{c}_{
    \begin{array}{c|c}
      A B & A \, \mathbf{r}_B
    \end{array}
  } =
  \left[
    \begin{array}{c|c}
      \mathbf{c}_A \, B &
      \mathbf{c}_A \, \mathbf{r}_B
    \end{array}
  \right]
\]

\[
  \mathbf{r}_{\text{control}} = \mathbf{r}_{
    \begin{array}{c}
      AB \\
      \midrule
      \mathbf{c}_A \, B
  \end{array}} =
  \left[
    \begin{array}{c}
      A \, \mathbf{r}_B \\
      \hline
      \mathbf{c}_A \, \mathbf{r}_B
    \end{array}
  \right]
\]

As we can see, the last element of each control checksum, is the same,
and also corresponds exactly to $C_{cr}\left[m+1,q+1\right]$.
This element is the one that allows us to detect errors
in the checksum blocks themselves, generalizing the approach to the entire
augmented result matrix and not just the upper left block containing the original product.

Wherever there is a mismatch between the returned checksum blocks and the associated computed control checksum,
we know: first, that there is an error in $C_{cr}$; and second, one coordinate of the error.

In the case of a single error inside $C_{cr}$,
there would be a single mismatch in both
$\mathbf{r}_{\text{control}}$ and $\mathbf{c}_{\text{control}}$.
The item indexes on the control checksum vectors
give the coordinates of the error inside the result matrix.

If there are multiple errors, they can be collinear or not.
By collinear errors we mean errors that share one coordinate,
or equivalently stated,
they are arranged on the same column or row in the result matrix.
Collinear errors can be individually detected and isolated.

The presence of at least two non collinear errors
can only be detected, but the exact coordinates of the individual errors
cannot be obtained.

\img{correctable}{.5\linewidth}{detectable errors cause exactly one mismatch in at least one of the control checksums.}

\img{uncorrectable}{.5\linewidth}{an error set that causes more than one mismatch in both checksums does not allow us to recover the individual error placement inside the matrix, since the configuration is ambiguous.}

\img{checkerr}{.5\linewidth}{errors in the checksum blocks themselves cause a mismatch in the shared item (the last item), and so can be detected as any other detectable error.}

\subsubsection{Error Correction}

We can only correct detectable errors, i.e. the ones we can isolate from the coordinates obtained by the checksum mismatches.

For a given error $C_e$ of coordinates $(i,j)$ in $C^* := C_{cr}$, we must choose one of the two control checksums and the associated checksum block in $C_{cr}$ to recover the original result value.
To do this, we simply enstablish along which axis the errors are collinear, and use the checksum of the opposite axis (the one which contains more than one mismatch) to compute the correction values.
For example, if the errors are arranged on a column, the column control checksum will contain a single mismatch while the row control checksum will contain as many mismatches as the error count, thus we have to use the row checksum because it contains distinct correction values for each error.
Of course, if there is only a single error present, one control checksum is as good as the other.
We can discard errors in the checksum blocks
as they don't corrupt the original multiplication result.
Actually, though, when using buffering strategies as discussed in \hyperref[sec:strategies]{Section 5}, we need the checksum blocks to be intact. Thus, if we detect errors in the checksum blocks, we can either correct them or we can
recalculate them after correcting the errors in the result block.
In our implementation we recalculate them for simplicity.

For column-collinear errors, the correction formula is:
\[
  C_{i,j} = C^*_{i,q+1} - \mathbf{r}_{\text{control}_{i}} + C_e
\]
For row collinear errors, the correction formula is:
\[
  C_{i,j} = C^*_{m+1,j} - \mathbf{c}_{\text{control}_{j}} + C_e
\]

These formulas are quite easy to derive by solving a system of two equations:
the equality between a control checksum and the associated checksum block in $C^*$ when no error is present,
and the actual control checksum computation equation where the error $C_e$ shows up.

\subsection{Strassen algorithm}

Strassen algorithm is an alternative to the traditional algorithm for matrix multiplications.
It leverages some mathematical properties, to reduce the number of products required to compute a given matrix product.

Although it is usually not described like this, the traditional algorithm can be seen as a series of products of submatrices.
Figure~\ref{img:matmul-2x2-normal} shows that, when splitting the operands into 4 blocks, 8 multiplications are required.
Instead, Strassen algorithm is able to reduce this number to 7 (as visible in figure~\ref{img:matmul-2x2-strassen}), switching the removed one with a series of less computational intensive sums.

At the beginning we believed that this algorithm could be useful to speed up our program, but then for multiple reasons described below we decided not to use it.

\img{matmul-2x2-normal}{.5\linewidth}{a product of 2x2 matrices with the traditional algorithm}
\img{matmul-2x2-strassen}{.6\linewidth}{A product of 2x2 matrices with the Strassen algorithm}

\newpage
\subsection{CUDA queues}

On top of the parallelization of the threads, CUDA also allows to parallelize different types of operations.
In particular, it defines three queues:
\begin{itemize}
	\itemsep 0em
	\item H2D: the queue where the memory transfers from host to device happen
	\item compute: where all threads are scheduled
	\item D2H: where the GPU schedules the memory transfers from device to host
\end{itemize}
If the code is well written, CUDA allows different operations to run concurrently, if they belong to different queues.
Ideally, this means for example that a kernel, a \texttt{cudaMemcpyHostToDevice}, and a \texttt{cudaMemcpyDeviceToHost} could all run concurrently.
On top of this, the compute queue can also allow multiple kernel being executed concurrently, if they belong to different streams and hardware resources are enough.
Instead, the H2D and D2H queues can only have a single data transfer each at the same time, regardless of streams.

This is however limited by the restrictions imposed by the scheduler:
\begin{itemize}
	\itemsep 0em
	\item An operation is only called if all the previous operations in the same stream are complete
	\item An operation is only called if all the previous ones in the same queue have been completed
	\item A blocked operation in a queue blocks all subsequent operations, even if they belong to different streams
	\item The signal that notifies from the compute queue to the D2H queue that a kernel is finished is delayed to when all sequential threads are finished
\end{itemize}
This means that calling operations in a breadth-first approach (eg. $kernel_{stream1} \rightarrow kernel_{stream 2} \rightarrow D2H_{stream 1} \rightarrow D2H_{stream 2}$) may lead to $D2H_{stream 1}$ having to wait for the completion of $kernel_{stream 2}$ before starting.
If instead operations are called with a depth-first approach (eg. $kernel_{stream1} \rightarrow D2H_{stream 1}  \rightarrow kernel_{stream 2}\rightarrow D2H_{stream 2}$), then the risk does not exist.

In our project, we paid attention to always schedule operations in a depth-first approach.

\subsection{Tiled matrix multiplication}

The tiled matrix multiplication is an algorithm that implements the traditional matrix multiplication algorithm, by exploiting the shared memory of a GPU to speed up the process.

\subsubsection{Without tiling}

The standard algorithm without tiling would have independent threads that proceed as follows, for each index $k$ required to compute $C_{rc}$:
\begin{itemize}
	\itemsep 0em
	\item Load $A_{rk}$ from global memory
	\item Load $B_{kc}$ from global memory
	\item Compute $A_{rk} \cdot B_{kc}$
	\item Add it to a local variable
	\item When the product is finished, store it to $C_{rc}$
\end{itemize}

When multiplying a matrix $m \times n$ by a matrix $n \times q$, this approach requires $2 \cdot m \cdot n \cdot q$ loads from global memory.

\subsubsection{With tiling}

With the tiling approach, we define \textit{tile} a square region of $T \times T$ cells.
Both the result matrix and the operands are split into tiles of the same size.
A tile of the result matrix is computed by a block of threads, leveraging shared memory.
Different tiles of the result matrix are however computed independently.

Among the thread block $C_{RC}$, the thread that has to compute $C_{rc}$ ($r$ and $c$ are indices local to the tile) performs the following for each tile $t$ necessary for the computation:
\begin{itemize}
	\itemsep 0em
	\item Loads from global memory to shared memory the value $A_{rc}$ of the tile $A_{Rt}$
	\item Loads from global memory to shared memory the value $B_{rc}$ of the tile $B_{tC}$
	\item Waits for the other threads to load all their values to shared memory
	\item Computes and sums to a local variable $A_{rk} \cdot B_{kc}$ for $k$ going from $0$ to $T$ (excluded), reading the values from the shared memory
	\item When the product is finished, stores it to $C_{rc}$
\end{itemize}

This way, the number of global memory reads is reduced by a factor of $T$, thanks to the fact that each thread in the block only loads $2$ values per tile.
It then relies on the other threads to load the other $2\cdot(T-1)$ that it needs.
