\documentclass[12pt]{article}

\usepackage{graphicx}
\usepackage[a4paper,width=150mm,top=25mm,bottom=25mm, bindingoffset=10mm]{geometry}
\usepackage{xcolor}
\usepackage[justification=centering]{caption}
\usepackage{hyperref}
\usepackage{amsmath}
\usepackage{booktabs}
\usepackage{enumitem}

\definecolor{light-gray}{gray}{0.9}
\newcommand{\code}[1]{\colorbox{light-gray}{\texttt{#1}}}

\newcommand{\img}[3]{
	\begin{figure}[h]
		\centering
		\includegraphics[width=#2]{images/#1}
		\caption{#3}\label{img:#1}
	\end{figure}
}

\title{Gigacheck}
\author{Emanuele Messina, Francesco Risso}
\date{2025}

\begin{document}

\maketitle

\begin{abstract}
	In this project, we developed and implemented
	a fault-tolerant matrix multiplication algorithm
	capable of handling matrices of arbitrary sizes on a GPU.
	The algorithm incorporates an error detection and correction technique
	to ensure computational accuracy in the presence of errors, and
	different buffering strategies were designed
	to manage memory constraints
	by splitting large matrices into manageable chunks
	that fit within GPU memory.
	We wrote a self-contained benchmark program
	to evaluate the algorithm performance
	when varying settings such as input size, available memory and buffering strategies.
\end{abstract}

\section{Introduction}

TODO spiegare perchè tutto

This paper is structured as follows:
\begin{itemize}
	\item Section 2 describes TODO ...
\end{itemize}

\section{Background Theory}
\label{sec:background}

\subsection{ABFT matrix multiplication}

The basic idea behind Algorithm-Based Fault Tolerant (ABFT) matrix multiplication
is to use checksums to verify the correctness of the computation.
The following paragraphs illustrate the steps we perform:

\subsubsection{Checksum Calculation}
Let \( A \) be an \( m \times n \) matrix,
\( B \) be an \( n \times q \) matrix,
and \( C \) be the resulting \( m \times q \) matrix
from the multiplication \( C = A B \).
Before performing the matrix multiplication,
we calculate the column checksum of \( A \) and the row checksum of \( B \):

\[
  \textnormal{Column checksum of } A: \quad
  \mathbf{c}_A =
  \left[
    \sum_{i=1}^{n} a_{i1},
    \sum_{i=1}^{n} a_{i2},
    \ldots, \sum_{i=1}^{n} a_{in}
  \right]
\]

\[
  \textnormal{Row checksum of } B: \quad
  \mathbf{r}_B = \left[ \sum_{j=1}^{n} b_{1j}, \sum_{j=1}^{n} b_{2j}, \ldots, \sum_{j=1}^{n} b_{nj} \right]^T
\]

\textit{
  The row checksum of a matrix is a column vector where each element
  is the sum of the elements in the corresponding row of the matrix.
  Similarly, the column checksum is a row vector where
each element is the sum of the elements in the corresponding column of the matrix.}

\subsubsection{Augmented Matrices}

We then create augmented matrices $A_{c}$ (\( m+1 \times n \)) and $B_{r}$ (\( n \times q+1 \))
by appending the column and row checksum vectors, respectively, to the original matrices:
Thus we get:
\[
  A_{c} =
  \left[
    \begin{array}{c}
      A \\
      \midrule
      \mathbf{c}_A
    \end{array}
  \right]
  \quad
  ,
  \quad
  B_{r} = \left[
    \begin{array}{c|c}
      B &
      \mathbf{r}_B
    \end{array}
  \right]
\]
\subsubsection{Multiplication}
We multiply $ A_{c} $ and $ B_{r} $ together, yielding
\[
  C_{cr} = A_{c} B_{r} = \left[
    \begin{array}{c|c}
      A B               & A \, \mathbf{r}_B            \\
      \hline
      \mathbf{c}_A \, B & \mathbf{c}_A \, \mathbf{r}_B
    \end{array}
  \right]
\]
We see that the original product is preserved
in the upper left block,
while the other blocks contain checksum information.

\subsubsection{Checksum Verification}
Considering the column and row checksums of the upper left block of
the resulting matrix \( C_{cr} \), the following properties hold:

\[
  \mathbf{c}_{AB} = \left[\mathbf{c}_A \, B \right]
  \quad
  ,
  \quad
  \mathbf{r}_{AB} = \left[A \, \mathbf{r}_B \right]
\]

The proof is trivial.

This is the property that we ultimately exploit to correct errors in the computation,
because if the upper block returned to us is corrupted,
then we can compute its checksums (we will call them control checksums)
and compare them against the peripheral blocks to at least detect the corruption of the result.

In general, the column checksum of the upper blocks of the augmented result
is equal to the last row of the augmented result, and the row cheksum of the left blocks of the augmented result
is equal to the last column the augmented result. In formulas:

\[
  \mathbf{c}_{\text{control}} = \mathbf{c}_{
    \begin{array}{c|c}
      A B & A \, \mathbf{r}_B
    \end{array}
  } =
  \left[
    \begin{array}{c|c}
      \mathbf{c}_A \, B &
      \mathbf{c}_A \, \mathbf{r}_B
    \end{array}
  \right]
\]

\[
  \mathbf{r}_{\text{control}} = \mathbf{r}_{
    \begin{array}{c}
      AB \\
      \midrule
      \mathbf{c}_A \, B
  \end{array}} =
  \left[
    \begin{array}{c}
      A \, \mathbf{r}_B \\
      \hline
      \mathbf{c}_A \, \mathbf{r}_B
    \end{array}
  \right]
\]

As we can see, the last element of each control checksum, is the same,
and also corresponds exactly to $C_{cr}\left[m+1,q+1\right]$.
This element is the one that allows us to detect errors
in the checksum blocks themselves, generalizing the approach to the entire
augmented result matrix and not just the upper left block containing the original product.

Wherever there is a mismatch between the returned checksum blocks and the associated computed control checksum,
we know: first, that there is an error in $C_{cr}$; and second, one coordinate of the error.

In the case of a single error inside $C_{cr}$,
there would be a single mismatch in both
$\mathbf{r}_{\text{control}}$ and $\mathbf{c}_{\text{control}}$.
The item indexes on the control checksum vectors
give the coordinates of the error inside the result matrix.

If there are multiple errors, they can be collinear or not.
By collinear errors we mean errors that share one coordinate,
or equivalently stated,
they are arranged on the same column or row in the result matrix.
Collinear errors can be individually detected and isolated.

The presence of at least two non collinear errors
can only be detected, but the exact coordinates of the individual errors
cannot be obtained.

\img{correctable}{.5\linewidth}{detectable errors cause exactly one mismatch in at least one of the control checksums.}

\img{uncorrectable}{.5\linewidth}{an error set that causes more than one mismatch in both checksums does not allow us to recover the individual error placement inside the matrix, since the configuration is ambiguous.}

\img{checkerr}{.5\linewidth}{errors in the checksum blocks themselves cause a mismatch in the shared item (the last item), and so can be detected as any other detectable error.}

\subsubsection{Error Correction}

We can only correct detectable errors, i.e. the ones we can isolate from the coordinates obtained by the checksum mismatches.

For a given error $C_e$ of coordinates $(i,j)$ in $C^* := C_{cr}$, we must choose one of the two control checksums and the associated checksum block in $C_{cr}$ to recover the original result value.
To do this, we simply enstablish along which axis the errors are collinear, and use the checksum of the opposite axis (the one which contains more than one mismatch) to compute the correction values.
For example, if the errors are arranged on a column, the column control checksum will contain a single mismatch while the row control checksum will contain as many mismatches as the error count, thus we have to use the row checksum because it contains distinct correction values for each error.
Of course, if there is only a single error present, one control checksum is as good as the other.
We can discard errors in the checksum blocks
as they don't corrupt the original multiplication result.
Actually, though, when using buffering strategies as discussed in \hyperref[sec:strategies]{Section 5}, we need the checksum blocks to be intact. Thus, if we detect errors in the checksum blocks, we can either correct them or we can
recalculate them after correcting the errors in the result block.
In our implementation we recalculate them for simplicity.

For column-collinear errors, the correction formula is:
\[
  C_{i,j} = C^*_{i,q+1} - \mathbf{r}_{\text{control}_{i}} + C_e
\]
For row collinear errors, the correction formula is:
\[
  C_{i,j} = C^*_{m+1,j} - \mathbf{c}_{\text{control}_{j}} + C_e
\]

These formulas are quite easy to derive by solving a system of two equations:
the equality between a control checksum and the associated checksum block in $C^*$ when no error is present,
and the actual control checksum computation equation where the error $C_e$ shows up.

\subsection{Strassen algorithm}

Strassen algorithm is an alternative to the traditional algorithm for matrix multiplications.
It leverages some mathematical properties, to reduce the number of products required to compute a given matrix product.

Although it is usually not described like this, the traditional algorithm can be seen as a series of products of submatrices.
Figure~\ref{img:matmul-2x2-normal} shows that, when splitting the operands into 4 blocks, 8 multiplications are required.
Instead, Strassen algorithm is able to reduce this number to 7 (as visible in figure~\ref{img:matmul-2x2-strassen}), switching the removed one with a series of less computational intensive sums.

At the beginning we believed that this algorithm could be useful to speed up our program, but then for multiple reasons described below we decided not to use it.

\img{matmul-2x2-normal}{.5\linewidth}{a product of 2x2 matrices with the traditional algorithm}
\img{matmul-2x2-strassen}{.6\linewidth}{A product of 2x2 matrices with the Strassen algorithm}
\subsection{CUDA queues}

\section{Overview of the Algorithm}
\label{sec:overview}

Given the matrices $A$ and $B$, our program processes them via to the following steps:
\begin{itemize}
	\itemsep 0em
	\item Choose if and how they need to be split
	\item For each block of the final matrix:
	      \begin{itemize}[topsep=0pt]
		      \itemsep 0em
		      \item Set $C$ to a matrix of zeros
		      \item For each couple of blocks from $A$ and $B$ that is required to compute the current block of $C$:
		            \begin{itemize}[topsep=0pt]
			            \itemsep 0em
			            \item Load the block of $A$
			            \item Load the block of $B$
			            \item Add $AB$ to the current $C$
			            \item Add errors to $C$
			            \item Find and correct the errors on $C$
		            \end{itemize}
		      \item Unload $C$ to the RAM
	      \end{itemize}
\end{itemize}

All the steps are described more in details in the following paragraphs.

\section{Block Product, Error Detection and Correction}
\label{sec:block}

\subsection{Computing checksums}

The checksums are computed by a single kernel, that based on a parameter computes row or column checksums.

For row checksums, a thread block is generated for each row, with a set of threads that initially sum a portion of the row, in a linear way.
After that, the different threads of the block organize themselves to sum all the partial sums in a dichotomous way.

Similarly, column checksums generate a thread block for each column, performing the same operations in the orthogonal direction.

After the computation, the kernel is able to store the checksum either in the last row or column of the matrix, or into a separate vector, in order to avoid overriding the previous checksums (the control checksums of C).

\img{checksums-pseudocode}{0.7\linewidth}{the pseudocode of the kernel that computes the checksums}

\subsection{Computing the product}

When the matrices are small enough to fit on GPU (or we are working on blocks), the multiplication is realized by means of the standard tiled algorithm for GPU.

Initially, we believed that we could apply one or two recursions of Strassen algorithm to reduce the number of products.
Then, we realized that this would make the same error appear in different submatrices: this would make it much harder to correct it, since non-collinear errors are not correctable.
For this reason, we decided to drop Strassen algorithm at this smaller size level.

\img{tiled-matmul-pseudocode}{0.7\linewidth}{the pseudocode of the kernel that computes the matrix multiplication using the tiled approach}

\subsection{Error addition}

In order to demonstrate the error correction capability, some errors can be added after the computation of the product.
The number of errors per product can be set by the user, as well as if they must be collinear or not.

The program randomly chooses a set of distinct elements where to add errors, and adds a random delta to the chosen element.

\subsection{Error correction}

After computing the product and adding the errors, checksums are computed again on the result, both on the rows and the columns.
A kernel is then run to find mismatches between the control checksums (the ones obtained as the sum of rows or columns of C) and the ones in C (obtained from the multiplication).

The mismatches are then compared to check if errors are present, and if they are collinear (otherwhise, they are not correctable).
If the errors are correctable, they are corrected by using the control checksums.
If they are not correctable, or there are errors on the checksums, then the caller is notified.

\img{edc-pseudocode}{0.7\linewidth}{the pseudocode of the function that detects and corrects the errors (\texttt{find\_mismatches} is a GPU kernel, while the rest is executed on CPU)}

\section{Buffering Strategies for Large Matrices}
\label{sec:strategies}


Whenever the matrices are too large for all the allocations to fit on GPU, then they are split into blocks.
The algorithm can choose two values:
\begin{itemize}
	\item \code{num\_split\_common\_dim}, the number of splits for A's rows and B's columns. When splitting in this direction, a single value of C will be the sum of this amount of products among blocks of A and B.
	\item\code{num\_split\_other\_dim}, the number of splits for A's columns and B's rows. This will result in C being made of a square grid of blocks, with this value as edge. The different blocks will be independent from each other.
\end{itemize}

An example of matrix splitting is provided in figure~\ref{img:example-of-splits}.

\img{example-of-splits}{.6\linewidth}{An example of how matrices A and B can be split}

In order to choose how to split, the algorithm computes the required memory, and compares it with the available memory.
If the available memory is smaller by a factor $k$, then:
\begin{itemize}
	\itemsep 0em
	\item \code{num\_split\_common\_dim} is assigned the value $\sqrt{k}$.
	\item\code{num\_split\_other\_dim} is assigned the value $\lceil \frac{k}{num\_split\_other\_dim} \rceil$.
\end{itemize}

\subsection{Strategy 1: no buffering}

This is the naive approach: the GPU memory is divided into 3 blocks: A, B and C.
At every product, the iteration loads A and B in parallel, then sums the product into C.
Whenever all the multiplications for a block of C are done, the block itself is unloaded to the CPU memory, while in the meantime the next A and B can start loading.

A timing diagram of this strategy can be seen in figure~\ref{}.

\subsection{Strategy 2: double buffer for A and B}

When using strategy 1, the multiplication requires A and B to be loaded: this means that while A and B are being transfered to GPU, no computation is being done (except for the checksums, that are however very fast).
Similarly, loading the next A and B requires the product to be completed, thus having no memory transfer while computing the product.
Effectively, this means that the 3 CUDA queues are mostly disjoint: the only overlaps are H2D and D2H when copying C (not at every iteration), and H2D and compute while calculating the checksums (for a very short time).

Strategy 2 aims to improve this situation, by dividing the GPU memory in two extra blocks (5 in total, called A, A', B, B' and C).
Initially, A and B are pre-loaded.
Then, before starting the product, A' and B' start to load the next block, in an asynchronous way.
At the same time, C can compute the product on A and B.
When both the new loading and the product are finished, the buffers are swapped: C can immediately start multiplying A' and B', while A and B load the next block.
This allows to overlap the H2D and compute queues in the time when the ``offline'' A and B are loading.

Figure~\ref{} shows the timing relative to this strategy.

\subsection{Strategy 3: double buffer for A, B and C}

This strategy brings to the extreme the idea behind strategy 2.
In strategy 2, multiplications are still forbidden while C is offloading, to avoid overriding not-yet-saved results.
Strategy 3 introduces a new buffer, called C', that works similarly to A' and B'.
At the beginning, the product is computed on C.
Then, when a full block has been computed, C is switched with C', to be able to immediately start the computation, while the offloading of C can go on in the background.
This ideally allows the compute stream (the bottleneck of the complex task of matrix multiplications) to run continuously, regardless of the status of the transfer queues.

Differently from strategy 2, however, the gained parallelism only occurs once every \texttt{num\_split\_common\_dim} iterations: this fact makes the speedup less impactful, with the risk of it being shadowed by the reduction in block sizes required to fit C' in the GPU memory, that can lead to more multiplications required.

Figure~\ref{} displays the timing diagram, in the optimal case when the addition of C' does not require more multiplications.

\subsection{Strategy 4: double concurrent product}

Since the slowest part of the program is the actual computation of the product, we tried to execute more than one in parallel.
Under the (uncertain) assumption that two multiplications can be done in parallel, this would really speed up the program.
Moreover, if the assumption is false, this idea should not be slower than strategy 3.

Strategy 4 uses the buffers C and C' to compute a block of the final matrix into C, and the block to its right into C', if it exists.
Using this method, the two products multiply the same block of A with shifted blocks of C: therefore, we can use just 3 buffers for the operands: A, B and B'.
This increases the size of each buffer, and reduces the required memory copies.

For this final strategy we draw the timing diagram for the two cases: if the parallel multiplications hypothesis holds (figure~\ref{}) or if it is wrong (figure~\ref{}).

\subsection{Strassen algorithm}

We initially discussed about adding an option to use Strassen algorithm at this higher level.
Ideally, it could be applied with any of the strategies described above.
The problem of this algorithm is that it requires 7 temporary matrices, for which we saw two options: storing all them in GPU memory, or loading and unloading them at need.

In order to compare the two options among them and with the traditional algorithm, we considered the case described in figure~\ref{img:matmul-2x2-normal}, where A and B do not fit on GPU memory, but they fit if they are divided into 4 blocks.

\img{matmul-2x2-normal}{.5\linewidth}{A product of 2x2 matrices with the traditional algorithm}
When using Strassen algorithm, the product would be as described in figure~\ref{img:matmul-2x2-strassen}.

As visible in figure~\ref{img:matmul-flow-normal}, the traditional algorithm requires 20 memory transfers, to load the blocks of A and B, and to offload C when computed.
This example assumes that the GPU is divided into 3 blocks of equal size (the 3 squares on top of each other, in the image)
\img{matmul-flow-normal}{\linewidth}{The series of load/calc/unload operations for a 2x2 product with the traditional algorithm. The colors indicate what has been loaded (red), computed (blue), offloaded (green) or left unchanged (black) from the previous step. An apex indicates that the result is partial.}
If we want to use Strassen algorithm with blocks of the same size, figure~\ref{img:matmul-flow-strassen-more-transfers} reveals that 45 memory transfers are required, because the program needs to offload temporary sums and matrices to make space for the other values.
\img{matmul-flow-strassen-more-transfers}{\linewidth}{The series of load/calc/unload operations for a 2x2 product with Strassen algorithm, if we want to have the GPU memory divided in just 3 blocks}
If instead we decide to have more blocks for the Strassen algorithm, in order to avoid offloading temporary results, we would need 21 memory transfers, as shown in figure~\ref{img:matmul-flow-strassen-more-memory}.
This removes one multiplication with respect to the traditional algorithm, at the cost of adding just an extra memory transfer.
In principle it would be good, if not for the fact that this approach requires the blocks to have half the size, thus having on average $\sqrt{2}$ more blocks.
That would then require more multiplications than what the Strassen algorithm saves, thus making its application not beneficial.
\img{matmul-flow-strassen-more-memory}{\linewidth}{The series of load/calc/unload operations for a 2x2 product with Strassen algorithm, if we want to avoid offloading temporary results}

As a conclusion, we realized that Strassen algorithm was an efficient way to save computation time, but only if the full matrices were fully available on GPU, thus making memory transfers not needed.

\section{Results}
\label{sec:results}

We wrote a simple CLI program to evaluate the algorithm under different conditions.
The program allows setting the input dimensions, the number of errors to introduce
and whether they should be collinear or not,
the simulated maximum amount of global memory, and the buffering strategy to use.
Several minor debugging parameters can be set, but are out of the scope of this paper.

We tested the four strategies on a MX130 laptop GPU, which unfortunately makes the results dependent on the current display workload, but we tried to reduce that for what it was possible.
Another thing to consider is that display GPUs have a timeout on kernel executions, so we couldn't run tests with very large matrices, and we believe that for bigger matrices the calculated performance should increase asymptotically.

For this GPU (Maxwell architecture), nvprof didn't have support for performance counters, and Nsight Compute suite was only available for the newer Ampere architecture.
For this reason we embedded some sort of profiling in the program itself.
As performance metrics, we used the overall execution time of the program (both total CPU and CUDA only), the total floating point arithmetic intensity and the average program performance (calculated as the total floating point operations over the total CUDA time).

To calculate the total floating point operations and the associated transfers to and from global memory (to then calculate the arithmetic intensity), we manually calculated the floating point operation and associated global memory transfers for each kernel we wrote. The program registers each kernel call and add its flops and transfers to their respective counter.

To calculate the effective compute time we use CUDA events to register the start and end of the program (getting the total CUDA time with respect to the default stream).
Then, still with CUDA events, we track each cudaMemcpy time (getting the total RAM-GPU transfer time), and we subtract it from the total CUDA time (because )

That said, we hereby present our considerations on the obtained results.


\end{document}
